%%!TEX encoding = UTF-8 Unicode 


\documentclass{beamer}
\usepackage[T2A]{fontenc}

%\usefont{T2A}{fta}{m}{sl}
\usepackage[utf8]{inputenc}
\usepackage[english,russian]{babel}

%\selectlanguage{russian}

%\usepackage{amssymb,amsfonts,amsmath,mathtext}
\usepackage{cite,enumerate,float,indentfirst}
\usetheme{Singapore}
\usecolortheme{default}
\title{Cryotec Service - система журналирования техобслуживания}
\author{Сергей Климов}
\date{Август 2010}

\begin{document}

%%титульная страница
\maketitle


\section{Описание функций}

\subsection{}
\begin{frame}
\frametitle{Функции программного продукта}
	\begin{itemize}
		\item Учет парка машин
		\item Учет клиентов
		\item Журналирование событий техобслуживания, ремонтов и неисправностей
		\item Создание отчетов
		\item Выборки по различным параметрам
		\item Прогнозирование времени предстоящих техобслуживаний на основе введенных
		данных
		\item Уведомление персонала о предстоящих событиях
		\item Управление правами пользователей системы
		\item Учет расходных материалов, используемых во время техобслуживаний
	\end{itemize}
\end{frame}


\begin{frame}
\frametitle{Учет парка машин}
	\begin{itemize}
		\item Каждая машина, введенная в систему, имеет:
		\begin{itemize}
			\item Запись о клиенте, которому продана/который ее использует
			\item Марку
			\item Тип (категорию)
			\item Данные о необходимом количестве наработанных моточасов для следующего
			техобслуживания
			\item Данные о текущем количестве моточасов
			\item Периодичность профосмотра (в месяцах)
		\end{itemize}
		\item Допускаются выборки по любым сочетаниям этих данных, а также составление
	отчетов
		\item К машине может быть прикреплено неограниченное количество файлов любых
		форматов (например, схемы, таблицы, фотографии)
	\end{itemize}
\end{frame}

\begin{frame}
\frametitle{Учет клиентов}
	О каждом клиенте компании хранится информация о реквизитах, машинах, которые
	клиент купил и/или использует. Возможны выборки клиентов, составление отчетов
	о них.
	Возможно прикрепление файлов к профилю клиента (например, электронной копии
	договора с ним)
\end{frame}

\begin{frame}
\frametitle{Журналирование событий}
	Система позволяет вводить ``события'' (техобслуживания, профосмотры, сообщения
	о неисправностях и ремонты), производить с ними операции выборки и составления
	отчетов.
\end{frame}

\begin{frame}
\frametitle{Журналирование неисправностей}
	Неисправность может быть введена в систему либо сотрудником компании через
	штатный интерфейс, либо клиентами компании через интерфейс для клиентов
	компании или путем посылки электронного письма на специальный адрес.

\end{frame}

\begin{frame}
\frametitle{Журналирование неисправностей}

	Система предоставляет следующие возможности в работе с неисправностями:
	\begin{itemize}
	  \item Просмотр ``незакрытых'' неисправностей (тех неисправностей, которые не
	  были ликвидированны ремонтом)
	  \item Ранжирование неисправностей по значимости (5 градаций по умолчанию)
	  \item Просмотр ремонтов, связанных с данной неисправностью
	  \item Просмотр техобслуживаний или профосмотров, в результате которых
	  выявлена данная неисправность (если неисправность не сообщена клиентом)
	  \item Выборка, составление отчетов
	  \item Прикрепление неограниченного количества файлов любых форматов к
	  неисправности (например, фотографий или таблиц)
	\end{itemize}
\end{frame}

\begin{frame}
\frametitle{Журналирование ремонтов}
	\begin{block}{Требования}
		\begin{itemize}
		  \item Ремонт должен быть привязан к неисправности
		  \item Ремонт может исправить или не исправить неисправность
	  	\end{itemize}
	\end{block}
	\begin{block}{Возможности, предоставляемые системой}
		\begin{itemize}
		  \item Выборки, составление отчетов
		  \item Прикрепление файлов
	  	\end{itemize}
	\end{block}
\end{frame}

\begin{frame}
\frametitle{Журналирование профосмотров}
	\begin{block}{Требования}
		\begin{itemize}
		  \item Профосмотр проводится в соответствии с требованиями о периодичности
		  техосмотра для данной машины
		  \item Во время профосмотра считывается значение отработанных моточасов с
		  машины
	  	\end{itemize}
	\end{block}
	\begin{block}{Возможности, предоставляемые системой}
		\begin{itemize}
		  \item Учет календарных дней, предупреждение персонала о предстоящих
		  профосмотрах
		  \item Выборки, составление отчетов
		  \item Прикрепление файлов

	  	\end{itemize}
	\end{block}
\end{frame}




\begin{frame}
\frametitle{Журналирование техобслуживаний}
	\begin{block}{Требования}
		\begin{itemize}
		  \item Техобслуживание проводится в соответствии с требованиями о
		  периодичности техобслуживания для конкретной машины и текущим значением
		  отработанных моточасов, считанных с нее во время последнего техобслуживания
		  или профосмотра
		  \item Во время техобслуживания считывается значение отработанных моточасов с
		  машины
		  \item Во время техобслуживания тратятся расходные материалы, специфичные для
		  марки конкретной машины
	  	\end{itemize}
	\end{block}
\end{frame}	
	
\begin{frame}
\frametitle{Журналирование техобслуживаний}
	\begin{block}{Возможности, предоставляемые системой}
		\begin{itemize}
		  \item Учет темпов ``накрутки'' моточасов на конкретной машине,
		  прогнозирование даты предстоящего техобслуживания и предупреждение персонала
		  о предстоящих профосмотрах
		  \item Выборки, составление отчетов
		  \item Прикрепление файлов
		  \item Учет расходных материалов используемых во время техосмотра,
		  планирование их использования

	  	\end{itemize}
	\end{block}
\end{frame}



\section{Технологии решения}

\begin{frame}
\frametitle{Выбор технологий}
	\begin{block}{Основные постулаты для выбора технологий}
		\begin{itemize}
			\item Клиент-серверная архитектура
			\item Нетребовательность к ресурсам аппаратного обеспечения клиента, расчет на
			работу на слабых компьютерах или даже на мобильных телефонах
			\item Отказоустойчивость
			\item Скорость разработки и простота поддержки
		\end{itemize}
	\end{block}
	\begin{block}{Выбранные технологии}
		\begin{itemize}
		  \item Python как основной язык программирования
		  \item Django как фреймворк для сервера
		  \item Pyjamas как фрейворк для клиента
		\end{itemize}
	\end{block}
\end{frame}


\subsection{Описание технологий}
\begin{frame}
\frametitle{Клиент-серверная архитектура}
	\begin{block}{Сервер}
	  Сервер запускается на мощностях компании и доступен постоянно. Этим
	  обеспечивается бесперебойность работы системы и целостность данных
	\end{block}
	\begin{block}{Клиент}
	  Клиенты (как сотрудники, так и клиенты системы) подключаются к серверу
	  различными способами для работы с системой
	\end{block}
\end{frame}	
	
	
	
\begin{frame}
\frametitle{Клиент-серверная архитектура: Клиент}	
	Клиенты могут подключиться к системе:
	  \begin{itemize}
	  	\item С помощью обычного браузера (например, Internet Explorer), который
	  имеется на любом компьютере и на многих мобильных телефонах
	  	\item С помощью специальной программы - клиента, запускаемого на компьютере
	  	клиента. Она имеет следующие возможности:
	  	\begin{itemize}
	  		\item Те же возможности, что и интерфейс браузера
	  		\item Оффлайн-работа. Т.е. работа в условиях, когда сервер компании
	  		недоступен - можно создать событие и сохранить его, затем, когда сервер
	  		будет доступен, произвести синхронизацию
	  	\end{itemize}
	  	\item Путем отсылки сообщения о неисправности через электронную почту
	  \end{itemize}
\end{frame}


\begin{frame}
\frametitle{Системные требования}
	\begin{block}{Сервер}
	  Сервер работает на любом достаточно современном оборудовании, доступ к
	  которому по сети имеют сотрудники компании
	\end{block}
	\begin{block}{Клиент}
	  Клиентская часть системы работает в любом современном браузере, необходим
	  лишь доступ по сети к серверу компании. В случае отсутсвия доступа
	  (например, на удаленном объекте), можно использовать специальную
	  программу-клиент, которая сохранит все данные для последующей синхронизацией
	  с сервером
	  
	  Возможно сообщение о неисправностях посредством обычного электронного письма
	\end{block}
\end{frame}


\begin{frame}
\frametitle{Отказоустойчивость}
	Используемые технологии делают данные системы удобными для резервного
	копирования. Персистентность данных в системе делает невозможным утрату данных в
	случае, например, аварийного отключения электроэнергии
\end{frame}

\begin{frame}
\frametitle{Скорость разработки и простота поддержки}
	Используемые технологии отличаются большим удобством разработки, и,
	соответственно, быстрым результатом
\end{frame}


\end{document}


